\documentclass{beamer}

\usepackage{amsmath}
\usepackage{fancyvrb}

\usetheme{Berlin}

\AtBeginSection[]%
{%
\begin{frame}%
  \begin{center}%
    \usebeamerfont{section title}\insertsection%
  \end{center}%
\end{frame}%
}%

\setbeamersize{text margin left=15pt}
\newcommand{\ca}[1]{{\color{blue}#1}}
\newcommand{\cb}[1]{{\color{violet}#1}}
\newcommand{\cc}[1]{{\color{red}#1}}
\newcommand{\cd}[1]{{\color{orange!90!black}#1}}
\newcommand{\ce}[1]{{\color{green!50!black}#1}}

\mode<presentation>

\title{Laws and Equations and Coq}

\author{Dave Laing}

\begin{document}

\begin{frame}
    \titlepage
\end{frame}

\begin{frame}
    \frametitle{Outline}
    \tableofcontents[pausesections]
\end{frame}

\section{Laws}

\begin{frame}
    \frametitle {Laws}

\begin{itemize}
\item Help when reasoning about code
\item Particularly code that makes use of typeclasses
\item There's only so much that a typeclass can enforce
\item For everything else there is
    \begin{itemize}
        \item Equational reasoning
        \item Theorem provers
        \item Angry letters to the authors
    \end{itemize}
\end{itemize}

\end{frame}

\begin{frame}
    \frametitle {Monoids are great!}
\begin{itemize}
\item Data.Foldable
\item Finger trees
\item Taking the tree out of tree shaped computations
\end{itemize}
\end{frame}

\begin{frame}[fragile]
    \frametitle {Monoids}

A monoid is a combination of 
\begin{itemize}
\item a type \verb?A?
\end{itemize}
and a function
\begin{itemize}
\item $\oplus$ \verb?:: A? $\rightarrow$ \verb?A? $\rightarrow$ \verb?A? 
\end{itemize}
that satisfies certain laws.

\end{frame}

\begin{frame}[fragile]
    \frametitle {Monoid Laws}

There must exist and element \verb?e :: A? such that for all \verb?a :: A?
\begin{columns}
\column{0.65\textwidth}
\begin{itemize}
\item \verb?e? $\oplus$ \verb?a = a?
\item \verb?a? $\oplus$ \verb?e = a?
\end{itemize}
\column{0.35\textwidth}
\begin{itemize}
\item[] (Left identity)
\item[] (Right identity)
\end{itemize}
\end{columns}

\vspace{10pt}

For all \verb?a, b, c :: A?
\begin{columns}
\column{0.65\textwidth}
\begin{itemize}
\item \verb?a? $\oplus$ \verb?(b? $\oplus$ \verb?c) = (a? $\oplus$ \verb?b)? $\oplus$ \verb?c?
\end{itemize}
\column{0.35\textwidth}
\begin{itemize}
\item[] (Associativity)
\end{itemize}
\end{columns}

\end{frame}

\begin{frame}[fragile]
    \frametitle {Monoid typeclass and List instance}

\begin{verbatim}
class Monoid a where
    mempty :: a
    mappend :: a -> a -> a
\end{verbatim}

\begin{verbatim}
instance Monoid [a] where
    mempty = []
    mappend = (++)
\end{verbatim}

(Remembering that \verb?(++)? has type \verb?[a] -> [a] -> [a]?)

\end{frame}

\begin{frame}
    \frametitle {Functors are great!}
\begin{itemize}
\item Apply a function to everything in a structure
\item Preserve the shape of the structure
\begin{itemize}
    \item You can double the values in a tree
\end{itemize}
\item Compose functions and apply once 
\item Saves multiple traversals
\begin{itemize}
    \item You can double and then add one to the values in a tree
\end{itemize}
\end{itemize}
\end{frame}

\begin{frame}[fragile]
    \frametitle {Aside: Type constructors}

Suppose \verb?F? is a type constructor (like Maybe or List)
\begin{itemize}
\item \Verb?F? isn't a type
\item \Verb?F Int?, \Verb?F String? are types
\item we'll use \Verb?F a?, \Verb?F b? as types for abstraction
\end{itemize}

\end{frame}

\begin{frame}[fragile]
    \frametitle {Functors}
A functor is a combination of 
\begin{itemize}
\item a type constructor \verb?F? 
\end{itemize}
and 
\begin{itemize}
\item a function \verb?fmap :: (a? $\rightarrow$ \verb?b)? $\rightarrow$ \verb?F a? $\rightarrow$ \verb?F b?
\end{itemize}
that satisfies certain laws.

\end{frame}

\begin{frame}[fragile]
    \frametitle {Functor Laws}

\begin{columns}
\column{0.75\textwidth}
\begin{itemize}
\item \verb?fmap id = id?
\item \verb?fmap g (fmap f xs) = fmap (g? $\circ$ \verb?f) xs?
\end{itemize}
\column{0.35\textwidth}
\begin{itemize}
\item[] (Identity)
\item[] (Composition)
\end{itemize}
\end{columns}

\end{frame}

\begin{frame}[fragile]
    \frametitle {Functor typeclass and List instance}

\begin{verbatim}
class Functor f where
    fmap :: (a -> b) -> f a -> f b 
\end{verbatim}

\begin{verbatim}
instance Functor [] where
    fmap = map
\end{verbatim}

(Remember that \verb?map? has type \verb?(a -> b) -> [a] -> [b]?)

\end{frame}

\section{Induction}

\begin{frame}[fragile]
    \frametitle {Natural numbers}
\begin{itemize}
    \item \verb?O :: Nat?
    \begin{itemize}
        \item this is zero
    \end{itemize}

    \item \verb?S :: Nat -> Nat?
    \begin{itemize}
        \item this is the successor function
        \item it returns one more than its input
    \end{itemize}

    \item \verb?Nat = O | S(x)?
    \begin{itemize}
        \item \verb?x? $\in$ \verb?Nat?
        \item so the definition of \verb?Nat? is recursive
    \end{itemize}

    \end{itemize}
\end{frame}

\begin{frame}[t,fragile]
    \frametitle {Addition}

\begin{overprint}

\onslide<1-2,6,13>
\begin{semiverbatim}
(+) :: Nat -> Nat -> Nat
O    + n = n
S(m) + n = S(m + n)
\end{semiverbatim}

\onslide<10>
\begin{semiverbatim}
(+) :: Nat -> Nat -> Nat
\ca{O}    + \cb{n} = n
S(m) + n = S(m + n)
\end{semiverbatim}

\onslide<11-12>
\begin{semiverbatim}
(+) :: Nat -> Nat -> Nat
\ca{O}    + \cb{n} = \cb{n}
S(m) + n = S(m + n)
\end{semiverbatim}

\onslide<3,7>
\begin{semiverbatim}
(+) :: Nat -> Nat -> Nat
O    + n = n
S(\ca{m}) + \cb{n} = S(m + n)
\end{semiverbatim}

\onslide<4-5,8-9>
\begin{semiverbatim}
(+) :: Nat -> Nat -> Nat
O    + n = n
S(\ca{m}) + \cb{n} = S(\ca{m} + \cb{n})
\end{semiverbatim}

\end{overprint}

\vspace{10pt}

\begin{center}
\begin{overprint}

\onslide<1>
\begin{semiverbatim}
S(S(O)) + S(S(S(O)))                      2 + 3
\end{semiverbatim}

\onslide<2-4>
\begin{semiverbatim}
S(\ca{S(O)}) + \cb{S(S(S(O)))}                      2 + 3
\end{semiverbatim}

\onslide<5>
\begin{semiverbatim}
S(\ca{S(O)}  + \cb{S(S(S(O)))})                 1+ (1 + 3)
\end{semiverbatim}

\onslide<6-8>
\begin{semiverbatim}
S(S(\ca{O})  + \cb{S(S(S(O)))})                 1+ (1 + 3)
\end{semiverbatim}

\onslide<9-11>
\begin{semiverbatim}
S(S(\ca{O}   + \cb{S(S(S(O)))}))             1+ 1+ (0 + 3)
\end{semiverbatim}

\onslide<12>
\begin{semiverbatim}
S(S      (\cb{S(S(S(O)))}))             1+ 1+ 3
\end{semiverbatim}

\onslide<13>
\begin{semiverbatim}
S(S      (S(S(S(O)))))                   5
\end{semiverbatim}

\end{overprint}
\end{center}

\end{frame}

\begin{frame}[t,fragile]
    \frametitle {Right identity - the long way}

Goal: $\forall$ \Verb?x :: Nat? \\
\quad \Verb?x + O = x?

\vspace{10pt}

\begin{overprint}

\onslide<1-2,6-7,11-16,20->
\begin{semiverbatim}
(+) :: Nat -> Nat -> Nat
O    + n = n
S(m) + n = S(m + n)
\end{semiverbatim}

\onslide<3>
\begin{semiverbatim}
(+) :: Nat -> Nat -> Nat
\ca{O}    + \cb{n} = n
S(m) + n = S(m + n)
\end{semiverbatim}

\onslide<4,5>
\begin{semiverbatim}
(+) :: Nat -> Nat -> Nat
\ca{O}    + \cb{n} = \cb{n}
S(m) + n = S(m + n)
\end{semiverbatim}

\onslide<8,17>
\begin{semiverbatim}
(+) :: Nat -> Nat -> Nat
O    + n = n
S(\ca{m}) + \cb{n} = S(m + n)
\end{semiverbatim}

\onslide<9-10,18-19>
\begin{semiverbatim}
(+) :: Nat -> Nat -> Nat
O    + n = n
S(\ca{m}) + \cb{n} = S(\ca{m} + \cb{n})
\end{semiverbatim}

\end{overprint}

\vspace{10pt}

\begin{center}
\begin{overprint}

\onslide<1>
\begin{semiverbatim}
O       + O = ?
\end{semiverbatim}

\onslide<2-4>
\begin{semiverbatim}
\ca{O}       + \cb{O} = ?
\end{semiverbatim}

\onslide<5>
\begin{semiverbatim}
O       + O = \cb{O}
\end{semiverbatim}

\onslide<6>
\begin{semiverbatim}
O       + O = O
S(O)    + O = ?
\end{semiverbatim}

\onslide<7-9>
\begin{semiverbatim}
O       + O = O
S(\ca{O})    + \cb{O} = ?
\end{semiverbatim}

\onslide<10>
\begin{semiverbatim}
O       + O = O
S(O)    + O = S(\ca{O} + \cb{O})
\end{semiverbatim}

\onslide<11-13>
\begin{semiverbatim}
\alert<12>{O       + O} = \alert<13>{O}
S(O)    + O = S(\alert<11-13>{O + O})
\end{semiverbatim}

\onslide<14>
\begin{semiverbatim}
O       + O = \alert{O}
S(O)    + O = S(\alert{O})
\end{semiverbatim}

\onslide<15>
\begin{semiverbatim}
O       + O = O
S(O)    + O = S(O)
S(S(O)) + O = ?
\end{semiverbatim}

\onslide<16-18>
\begin{semiverbatim}
O       + O = O
S(O)    + O = S(O)
S(\ca{S(O)}) + \cb{O} = ?
\end{semiverbatim}

\onslide<19>
\begin{semiverbatim}
O       + O = O
S(O)    + O = S(O)
S(S(O)) + O = S(\ca{S(O)} + \cb{O})
\end{semiverbatim}

\onslide<20-22>
\begin{semiverbatim}
O       + O = O
\alert<21>{S(O)    + O} = \alert<22>{S(O)}
S(S(O)) + O = S(\alert<20-22>{S(O) + O})
\end{semiverbatim}

\onslide<23>
\begin{semiverbatim}
O       + O = O
S(O)    + O = \alert{S(O)}
S(S(O)) + O = S(\alert{S(O)})
\end{semiverbatim}

\onslide<24->
\begin{semiverbatim}
O       + O = O
S(O)    + O = S(O)
S(S(O)) + O = S(S(O))
\end{semiverbatim}

\end{overprint}

\vspace{10pt}

\onslide<25>
The rest of \Verb?Nat? left as an exercise...

\end{center}

\end{frame}

\begin{frame}[fragile]
    \frametitle {Right identity - the pattern}

\begin{itemize}
\item Proof for \Verb?S(O)? used proof for \Verb?O?
\item Proof for \Verb?S(S(O))? used proof for \Verb?S(O)?
\item At each step we look at the last one and think "just one more"
\item Exploit the pattern, use a little abstraction.
\end{itemize}

\end{frame}

\begin{frame}[fragile]
    \frametitle {Right identity - the details}

\begin{itemize}
\item Show that if \Verb?x + O = x? then \Verb?S(x) + O = S(x)? 
    \begin{itemize}
        \item Captures the "just one more" part of the proof
    \end{itemize}
\item Need to prove the base case as well.
    \begin{itemize}
    \item In this case the base case is \Verb?O + O = O?
    \item So the "just one more" part has somewhere sensible to start from.
    \item Otherwise we can also prove that \Verb?x + S(O) = x? for $x > O$
    \end{itemize}
\end{itemize}

\end{frame}


\begin{frame}[fragile]
    \frametitle {Induction}

To prove \Verb?P(n)? for all \Verb?n? $\in$ \Verb?Nat?
\begin{itemize}
\item prove that \Verb?P(O)? holds
\item prove that \Verb?P(n)? $\rightarrow$ \Verb?P(S n)? holds
\end{itemize}

\vspace{10pt}

To prove \Verb?P(xs)? for all \Verb?xs? $\in$ \Verb?[a]?
\begin{itemize}
\item prove that \Verb?P([])? holds
\item prove that \Verb?P(xs)? $\rightarrow$ \Verb?P(x:xs)?
\end{itemize}

\vspace{10pt}

Similar for trees, any other recursive datatypes

\end{frame}

\section{Equational Reasoning}

\begin{frame}[t,fragile]
    \frametitle {Monoid - Left Identity}

\begingroup
\color{gray}\fontsize{10}{9.8}\selectfont

Goal : $\forall$ \verb?bs :: [a]? \\
    \quad \alert<2>{\Verb?[] ++ bs = bs?}

\vspace{10pt}

\begin{overprint}

\onslide<1-4,8->
\begin{semiverbatim}
(++) :: [a] -> [a] -> [a]
[]     ++ ys = ys
(x:xs) ++ ys = x:(xs ++ ys)
\end{semiverbatim}

\onslide<5>
\begin{semiverbatim}
(++) :: [a] -> [a] -> [a]
\ca{[]}     ++ \cb{ys} = ys
(x:xs) ++ ys = x:(xs ++ ys)
\end{semiverbatim}

\onslide<6-7>
\begin{semiverbatim}
(++) :: [a] -> [a] -> [a]
\ca{[]}     ++ \cb{ys} = \cb{ys}
(x:xs) ++ ys = x:(xs ++ ys)
\end{semiverbatim}

\end{overprint}

\endgroup

\vspace{10pt}

\begin{overprint}

\onslide<1-8>
Case: \ce{\Verb?[]?}

\onslide<9>
Case: \ce{\Verb?[]?} \checkmark

\end{overprint}

\vspace{5pt}

\begin{center}
\begin{overprint}

\onslide<3>
\begin{semiverbatim}
[] ++ bs = bs
\end{semiverbatim}

\onslide<4-6>
\begin{semiverbatim}
\ca{[]} ++ \cb{bs} = bs
\end{semiverbatim}

\onslide<7>
\begin{semiverbatim}
      \cb{bs} = bs
\end{semiverbatim}

\onslide<8-9>
\begin{semiverbatim}
      \alert<8>{bs} = \alert<8>{bs}
\end{semiverbatim}

\end{overprint}
\end{center}

\end{frame}

\begin{frame}[t,fragile]
    \frametitle {Monoid - Right Identity}

\begingroup
\color{gray}\fontsize{10}{9.8}\selectfont

Goal : $\forall$ \verb?bs :: [a]? \\
    \quad \alert<3,13>{\Verb?bs ++ [] = bs?}

\vspace{10pt}

\begin{overprint}

\onslide<1-6,10-16,20->
\begin{semiverbatim}
(++) :: [a] -> [a] -> [a]
[]     ++ ys = ys
(x:xs) ++ ys = x:(xs ++ ys)
\end{semiverbatim}

\onslide<7>
\begin{semiverbatim}
(++) :: [a] -> [a] -> [a]
\ca{[]}     ++ \cb{ys} = ys
(x:xs) ++ ys = x:(xs ++ ys)
\end{semiverbatim}

\onslide<8-9>
\begin{semiverbatim}
(++) :: [a] -> [a] -> [a]
\ca{[]}     ++ \cb{ys} = \cb{ys}
(x:xs) ++ ys = x:(xs ++ ys)
\end{semiverbatim}

\onslide<17>
\begin{semiverbatim}
(++) :: [a] -> [a] -> [a]
[]     ++ ys = ys
(\ca{x}:\cb{xs}) ++ \cc{ys} = x:(xs ++ ys)
\end{semiverbatim}

\onslide<18-19>
\begin{semiverbatim}
(++) :: [a] -> [a] -> [a]
[]     ++ ys = ys
(\ca{x}:\cb{xs}) ++ \cc{ys} = \ca{x}:(\cb{xs} ++ \cc{ys})
\end{semiverbatim}

\end{overprint}

\endgroup

\vspace{10pt}

\begin{overprint}

\onslide<1>
Cases:\quad\Verb?bs = []? \quad and \Verb?bs = (x:xs)?

\onslide<2-10>
Case:\quad{\ce{\Verb?bs = ?\alert<4>{\Verb?[]?}}} 

\onslide<11>
Case:\quad{\ce{\Verb?bs = []?}} \checkmark 

\onslide<12-24>
Case:\quad{\ce{\Verb?bs = ?\alert<14>{\Verb?(x:xs)?}}} \\
Assume: \alert<21>{\Verb?xs ++ []?}\Verb? = ?\alert<22-23>{\Verb?xs?}

\onslide<25>
Case:\quad{\ce{\Verb?bs = (x:xs)?}} \checkmark \\
Assume: \Verb?xs ++ [] = xs?

\end{overprint}

\vspace{5pt}

\begin{center}
\begin{overprint}

\onslide<3-4>
\begin{semiverbatim}
\alert<4>{bs} ++ [] = \alert<4>{bs}
\end{semiverbatim}

\onslide<5>
\begin{semiverbatim}
[] ++ [] = []
\end{semiverbatim}

\onslide<6-8>
\begin{semiverbatim}
\ca{[]} ++ \cb{[]} = []
\end{semiverbatim}

\onslide<9>
\begin{semiverbatim}
      \cb{[]} = []
\end{semiverbatim}

\onslide<10>
\begin{semiverbatim}
      \alert{[]} = \alert{[]}
\end{semiverbatim}

\onslide<11>
\begin{semiverbatim}
      [] = []
\end{semiverbatim}

\onslide<13-14>
\begin{semiverbatim}
\alert<14>{bs}     ++ [] = \alert<14>{bs}
\end{semiverbatim}

\onslide<15>
\begin{semiverbatim}
(x:xs) ++ [] = (x:xs)
\end{semiverbatim}

\onslide<16-18>
\begin{semiverbatim}
(\ca{x}:\cb{xs}) ++ \cc{[]} = (x:xs)
\end{semiverbatim}

\onslide<19>
\begin{semiverbatim}
\ca{x}:(\cb{xs} ++ \cc{[]}) = (x:xs)
\end{semiverbatim}

\onslide<20-22>
\begin{semiverbatim}
x:(\alert{xs ++ []}) = (x:xs)
\end{semiverbatim}

\onslide<23>
\begin{semiverbatim}
x:(\alert{xs})       = (x:xs)
\end{semiverbatim}

\onslide<24>
\begin{semiverbatim}
\alert{(x:xs)}       = \alert{(x:xs)}
\end{semiverbatim}

\onslide<25>
\begin{semiverbatim}
(x:xs)       = (x:xs)
\end{semiverbatim}

\end{overprint}
\end{center}

\end{frame}

\begin{frame}[t,fragile]
    \frametitle {Monoid - Associativity}

\begingroup
\color{gray}\fontsize{10}{9.8}\selectfont

Goal : $\forall$ \verb?bs, cs, ds :: [a]? \\
    \quad \alert<3,17>{\Verb?bs ++ (cs ++ ds) = (bs ++ cs) ++ ds?}

\vspace{10pt}

\begin{overprint}

\onslide<1-6,10,14-20,24,28,32->
\begin{semiverbatim}
(++) :: [a] -> [a] -> [a]
[]     ++ ys = ys
(x:xs) ++ ys = x:(xs ++ ys)
\end{semiverbatim}

\onslide<7,11>
\begin{semiverbatim}
(++) :: [a] -> [a] -> [a]
\ca{[]}     ++ \cb{ys} = ys
(x:xs) ++ ys = x:(xs ++ ys)
\end{semiverbatim}

\onslide<8-9,12-13>
\begin{semiverbatim}
(++) :: [a] -> [a] -> [a]
\ca{[]}     ++ \cb{ys} = \cb{ys}
(x:xs) ++ ys = x:(xs ++ ys)
\end{semiverbatim}

\onslide<21,25,29>
\begin{semiverbatim}
(++) :: [a] -> [a] -> [a]
[]     ++ ys = ys
(\ca{x}:\cb{xs}) ++ \cc{ys} = x:(xs ++ ys)
\end{semiverbatim}

\onslide<22-23,26-27,30-31>
\begin{semiverbatim}
(++) :: [a] -> [a] -> [a]
[]     ++ ys = ys
(\ca{x}:\cb{xs}) ++ \cc{ys} = \ca{x}:(\cb{xs} ++ \cc{ys})
\end{semiverbatim}

\end{overprint}

\endgroup

\vspace{10pt}

\begin{overprint}

\onslide<1>
Cases:\quad\Verb?bs = []? \quad and \Verb?bs = (x:xs)?

\onslide<2-14>
Case:\quad{\ce{\Verb?bs = ?\alert<4>{\Verb?[]?}}} 

\onslide<15>
Case:\quad{\ce{\Verb?bs = []?}} \checkmark 

\onslide<16-36>
Case:\quad{\ce{\Verb?bs = ?\alert<18>{\Verb?(x:xs)?}}} \\
Assume: \alert<33>{\Verb?xs ++ (cs ++ ds)?}\Verb? = ?\alert<34-35>{\Verb?(xs ++ cs) ++ ds?}

\onslide<37>
Case:\quad{\ce{\Verb?bs = (x:xs)?}} \checkmark \\
Assume: \Verb?xs ++ (cs ++ ds) = (xs ++ cs) ++ ds?

\end{overprint}

\vspace{5pt}

\begin{center}
\begin{overprint}

\onslide<3-4>
\begin{semiverbatim}
\alert<4>{bs} ++ (cs ++ ds) = (\alert<4>{bs} ++ cs) ++ ds
\end{semiverbatim}

\onslide<5>
\begin{semiverbatim}
[] ++ (cs ++ ds) = ([] ++ cs) ++ ds
\end{semiverbatim}

\onslide<6-8>
\begin{semiverbatim}
\ca{[]} ++ \cb{(cs ++ ds)} = ([] ++ cs) ++ ds
\end{semiverbatim}

\onslide<9>
\begin{semiverbatim}
       \cb{cs ++ ds}  = ([] ++ cs) ++ ds
\end{semiverbatim}

\onslide<10-12>
\begin{semiverbatim}
       cs ++ ds  = (\ca{[]} ++ \cb{cs}) ++ ds
\end{semiverbatim}

\onslide<13>
\begin{semiverbatim}
       cs ++ ds  = (\cb{cs})       ++ ds
\end{semiverbatim}

\onslide<14-15>
\begin{semiverbatim}
       \alert<14>{cs ++ ds}  =  \alert<14>{cs        ++ ds}
\end{semiverbatim}

\onslide<17-18>
\begin{semiverbatim}
    \alert<18>{bs} ++ (cs ++ ds) =     (\alert<18>{bs} ++ cs) ++ ds
\end{semiverbatim}

\onslide<19>
\begin{semiverbatim}
(x:xs) ++ (cs ++ ds) = ((x:xs) ++ cs) ++ ds
\end{semiverbatim}

\onslide<20-22>
\begin{semiverbatim}
(\ca{x}:\cb{xs}) ++ \cc{(cs ++ ds)} = ((x:xs) ++ cs) ++ ds
\end{semiverbatim}

\onslide<23>
\begin{semiverbatim}
\ca{x}:(\cb{xs} ++ \cc{(cs ++ ds)}) = ((x:xs) ++ cs) ++ ds
\end{semiverbatim}

\onslide<24-26>
\begin{semiverbatim}
x:(xs ++ (cs ++ ds)) = ((\ca{x}:\cb{xs}) ++ \cc{cs}) ++ ds
\end{semiverbatim}

\onslide<27>
\begin{semiverbatim}
x:(xs ++ (cs ++ ds)) = (\ca{x}:(\cb{xs} ++ \cc{cs})) ++ ds
\end{semiverbatim}

\onslide<28-30>
\begin{semiverbatim}
x:(xs ++ (cs ++ ds)) = (\ca{x}:\cb{(xs ++ cs)}) ++ \cc{ds}
\end{semiverbatim}

\onslide<31>
\begin{semiverbatim}
x:(xs ++ (cs ++ ds)) = \ca{x}:(\cb{(xs ++ cs)} ++ \cc{ds})
\end{semiverbatim}

\onslide<32-34>
\begin{semiverbatim}
x:(\alert{xs ++ (cs ++ ds)}) = x:((xs ++ cs) ++ ds)
\end{semiverbatim}

\onslide<35>
\begin{semiverbatim}
x:(\alert{(xs ++ cs) ++ ds)} = x:((xs ++ cs) ++ ds)
\end{semiverbatim}

\onslide<36>
\begin{semiverbatim}
\alert{x:((xs ++ cs) ++ ds)} = \alert{x:((xs ++ cs) ++ ds)}
\end{semiverbatim}

\onslide<37>
\begin{semiverbatim}
x:((xs ++ cs) ++ ds) = x:((xs ++ cs) ++ ds)
\end{semiverbatim}

\end{overprint}
\end{center}

\end{frame}

\begin{frame}[fragile]
    \frametitle {Monoid - Associativity - Bird style}
\begin{semiverbatim}
    bs ++ (cs ++ ds) = (bs ++ cs) ++ ds
                  \ce{bs = x:xs}
(x:xs) ++ (cs ++ ds) = ((x:xs) ++ cs) ++ ds
              \ce{second rule for ++}
x:(xs ++ (cs ++ ds)) = ((x:xs) ++ cs) ++ ds
              \ce{second rule for ++}
x:(xs ++ (cs ++ ds)) = (x:(xs ++ cs)) ++ ds
              \ce{second rule for ++}
x:(xs ++ (cs ++ ds)) = x:((xs ++ cs) ++ ds)
             \ce{inductive hypothesis}
x:((xs ++ cs) ++ ds) = x:((xs ++ cs) ++ ds)
\end{semiverbatim}
\end{frame}

\begin{frame}[t,fragile]
    \frametitle {Functor - Identity}

\begingroup
\color{gray}\fontsize{10}{9.8}\selectfont

Goal : $\forall$ \verb?bs :: [a]? \\
    \quad \alert<3,13>{\Verb?map id bs = bs?}

\vspace{10pt}

\begin{overprint}

\onslide<1-6,10-16,20->
\begin{semiverbatim}
map :: (a -> b) -> [a] -> [b]
map f [] = []
map f (x:xs) = f x : map f xs
\end{semiverbatim}

\onslide<7>
\begin{semiverbatim}
map :: (a -> b) -> [a] -> [b]
map \ca{f} \cb{[]} = []
map f (x:xs) = f x : map f xs
\end{semiverbatim}

\onslide<8-9>
\begin{semiverbatim}
map :: (a -> b) -> [a] -> [b]
map \ca{f} \cb{[]} = \cb{[]}
map f (x:xs) = f x : map f xs
\end{semiverbatim}

\onslide<17>
\begin{semiverbatim}
map :: (a -> b) -> [a] -> [b]
map f [] = []
map \ca{f} (\cb{x}:\cc{xs}) = f x : map f xs
\end{semiverbatim}

\onslide<18-19>
\begin{semiverbatim}
map :: (a -> b) -> [a] -> [b]
map f [] = []
map \ca{f} (\cb{x}:\cc{xs}) = \ca{f} \cb{x} : map \ca{f} \cc{xs}
\end{semiverbatim}

\end{overprint}

\endgroup

\vspace{10pt}

\begin{overprint}

\onslide<1>
Cases:\quad\Verb?bs = []? \quad and \Verb?bs = (x:xs)?

\onslide<2-10>
Case:\quad{\ce{\Verb?bs = ?\alert<4>{\Verb?[]?}}} 

\onslide<11>
Case:\quad{\ce{\Verb?bs = []?}} \checkmark 

\onslide<12-26>
Case:\quad{\ce{\Verb?bs = ?\alert<14>{\Verb?(x:xs)?}}} \\
Assume: \alert<23>{\Verb?map id xs?}\Verb? = ?\alert<24-25>{\Verb?xs?}

\onslide<27>
Case:\quad{\ce{\Verb?bs = (x:xs)?}} \checkmark \\
Assume: \Verb?map id xs = xs?

\end{overprint}

\vspace{5pt}

\begin{center}
\begin{overprint}

\onslide<3-4>
\begin{semiverbatim}
map id \alert<4>{bs} = \alert<4>{bs}
\end{semiverbatim}

\onslide<5>
\begin{semiverbatim}
map id [] = []
\end{semiverbatim}

\onslide<6-8>
\begin{semiverbatim}
map \ca{id} \cb{[]} = []
\end{semiverbatim}

\onslide<9>
\begin{semiverbatim}
       \cb{[]} = []
\end{semiverbatim}

\onslide<10-11>
\begin{semiverbatim}
       \alert<10>{[]} = \alert<10>{[]}
\end{semiverbatim}

\onslide<13-14>
\begin{semiverbatim}
map id     \alert<14>{bs}      = \alert<14>{bs}
\end{semiverbatim}

\onslide<15>
\begin{semiverbatim}
map id (x : xs)    = (x : xs)
\end{semiverbatim}

\onslide<16-18>
\begin{semiverbatim}
map \ca{id} (\cb{x} : \cc{xs})    = (x : xs)
\end{semiverbatim}

\onslide<19>
\begin{semiverbatim}
\ca{id} \cb{x} : map \ca{id} \cc{xs}   = (x : xs)
\end{semiverbatim}

\onslide<20>
\begin{semiverbatim}
\alert{id x} : map id xs   = (x : xs)
\end{semiverbatim}

\onslide<21>
\begin{semiverbatim}
\alert{x}    : map id xs   = (x : xs)
\end{semiverbatim}

\onslide<22-24>
\begin{semiverbatim}
x    : \alert{map id xs}   = (x : xs)
\end{semiverbatim}

\onslide<25>
\begin{semiverbatim}
x    : \alert{xs}          = (x : xs)
\end{semiverbatim}

\onslide<26-27>
\begin{semiverbatim}
\alert<26>{(x    : xs)}        = \alert<26>{(x : xs)}
\end{semiverbatim}

\end{overprint}
\end{center}

\end{frame}

\begin{frame}[t,fragile]
    \frametitle {Functor - Composition}

\begingroup
\color{gray}\fontsize{10}{9.8}\selectfont

Goal : $\forall$ \verb?bs :: [a], f :: a? $\rightarrow$ \verb?b, g :: b? $\rightarrow$ \verb?c? \\ 
    \quad \alert<3,21>{\Verb?map g (map f bs) = map (g . f) xs?}

\vspace{10pt}

\begin{overprint}

\onslide<1-6,10,14,18-24,28,32,36->
\begin{semiverbatim}
map :: (a -> b) -> [a] -> [b]
map f [] = []
map f (x:xs) = f x : map f xs
\end{semiverbatim}

\onslide<7,11,15>
\begin{semiverbatim}
map :: (a -> b) -> [a] -> [b]
map \ca{f} \cb{[]} = []
map f (x:xs) = f x : map f xs
\end{semiverbatim}

\onslide<8-9,12-13,16-17>
\begin{semiverbatim}
map :: (a -> b) -> [a] -> [b]
map \ca{f} \cb{[]} = \cb{[]}
map f (x:xs) = f x : map f xs
\end{semiverbatim}

\onslide<25,29,33>
\begin{semiverbatim}
map :: (a -> b) -> [a] -> [b]
map f [] = []
map \ca{f} (\cb{x}:\cc{xs}) = f x : map f xs
\end{semiverbatim}

\onslide<26-27,30-31,34-35>
\begin{semiverbatim}
map :: (a -> b) -> [a] -> [b]
map f [] = []
map \ca{f} (\cb{x}:\cc{xs}) = \ca{f} \cb{x} : map \ca{f} \cc{xs}
\end{semiverbatim}

\end{overprint}

\endgroup

\vspace{10pt}

\begin{overprint}

\onslide<1>
Cases:\quad\Verb?bs = []? \quad and \Verb?bs = (x:xs)?

\onslide<2-18>
Case:\quad{\ce{\Verb?bs = ?\alert<4>{\Verb?[]?}}} 

\onslide<19>
Case:\quad{\ce{\Verb?bs = []?}} \checkmark 

\onslide<20-42>
Case:\quad{\ce{\Verb?bs = ?\alert<22>{\Verb?(x:xs)?}}} \\
Assume:\quad\alert<39>{\Verb?map g (map f xs)?}\Verb? = ?\alert<40-41>{\Verb?map (g . f) xs?}

\onslide<43>
Case:\quad{\ce{\Verb?bs = (x:xs)?}} \checkmark \\
Assume:\quad\Verb?map g (map f xs) = map (g . f) xs ?

\end{overprint}

\vspace{5pt}

\begin{center}
\begin{flushleft}
\begin{overprint}

\onslide<3-4>
\begin{semiverbatim}
map g (map f \alert<4>{bs}) = map (g . f) \alert<4>{bs} 
\end{semiverbatim}

\onslide<5>
\begin{semiverbatim}
map g (map f []) = map (g . f) [] 
\end{semiverbatim}

\onslide<6-8>
\begin{semiverbatim}
map g (map f []) = map \ca{(g . f)} \cb{[]}
\end{semiverbatim}

\onslide<9>
\begin{semiverbatim}
map g (map f []) =             \cb{[]}
\end{semiverbatim}

\onslide<10-12>
\begin{semiverbatim}
map g (map \ca{f} \cb{[]}) =             []
\end{semiverbatim}

\onslide<13>
\begin{semiverbatim}
map g         \cb{[]} =             []
\end{semiverbatim}

\onslide<14-16>
\begin{semiverbatim}
map \ca{g}         \cb{[]} =             []
\end{semiverbatim}

\onslide<17>
\begin{semiverbatim}
              \cb{[]} =             []
\end{semiverbatim}

\onslide<18-19>
\begin{semiverbatim}
              \alert<18>{[]} =             \alert<18>{[]}
\end{semiverbatim}

\onslide<21-22>
\begin{semiverbatim}
      map g (map f \alert<22>{bs})       = map (g . f) \alert<22>{bs}
\end{semiverbatim}

\onslide<23>
\begin{semiverbatim}
      map g (map f (x:xs))   = map (g . f) (x:xs)
\end{semiverbatim}

\onslide<24-26>
\begin{semiverbatim}
      map g (map f (x:xs))   = map \ca{(g . f)} (\cb{x}:\cc{xs})
\end{semiverbatim}

\onslide<27>
\begin{semiverbatim}
      map g (map f (x:xs))   = \ca{(g . f)} \cb{x} : map \ca{(g . f)} \cc{xs}
\end{semiverbatim}

\onslide<28-30>
\begin{semiverbatim}
      map g (map \ca{f} (\cb{x}:\cc{xs}))   = (g . f) x : map (g . f) xs
\end{semiverbatim}

\onslide<31>
\begin{semiverbatim}
      map g (\ca{f} \cb{x} : map \ca{f} \cc{xs}) = (g . f) x : map (g . f) xs
\end{semiverbatim}

\onslide<32-34>
\begin{semiverbatim}
      map \ca{g} (\cb{f x} : \cc{map f xs}) = (g . f) x : map (g . f) xs
\end{semiverbatim}

\onslide<35>
\begin{semiverbatim}
  \ca{g} (\cb{f x}) : map \ca{g} (\cc{map f xs}) = (g . f) x : map (g . f) xs
\end{semiverbatim}

\onslide<36>
\begin{semiverbatim}
  \alert{g (f x)} : map g (map f xs) = (g . f) x : map (g . f) xs
\end{semiverbatim}

\onslide<37>
\begin{semiverbatim}
\alert{(g . f) x} : map g (map f xs) = (g . f) x : map (g . f) xs
\end{semiverbatim}

\onslide<38-40>
\begin{semiverbatim}
(g . f) x : \alert{map g (map f xs)} = (g . f) x : map (g . f) xs
\end{semiverbatim}

\onslide<41>
\begin{semiverbatim}
(g . f) x : \alert{map (g . f) xs}   = (g . f) x : map (g . f) xs
\end{semiverbatim}

\onslide<42-43>
\begin{semiverbatim}
\alert<42>{(g . f) x : map (g . f) xs}   = \alert<42>{(g . f) x : map (g . f) xs}
\end{semiverbatim}

\end{overprint}
\end{flushleft}
\end{center}

\end{frame}

\section{Coq}

\begin{frame}
    \frametitle {To the Coqmobile!}
\end{frame}

\begin{frame}[t,fragile]
    \frametitle {Coq extraction - append}
\begin{verbatim}
module Append where

import qualified Prelude

data List a =
   Nil
 | Cons a (List a)

append :: (List a1) -> (List a1) -> List a1
append xs ys =
  case xs of {
   Nil -> ys;
   Cons h t -> Cons h (append t ys)}
\end{verbatim}
\end{frame}

\begin{frame}[t,fragile]
    \frametitle {Coq extraction - map}
\begin{verbatim}
module Map where

import qualified Prelude

data List a =
   Nil
 | Cons a (List a)

map :: (a1 -> a2) -> (List a1) -> List a2
map f l =
  case l of {
   Nil -> Nil;
   Cons a t -> Cons (f a) (map f t)}
\end{verbatim}
\end{frame}

\section*{}

\begin{frame}
    \frametitle{Conclusion}
\begin{itemize}
\item Equational reasoning is easy and useful
\item Theorem provers can help if you want to be really sure
\item Very few excuses for law-breaking typeclass instances
\end{itemize}
\end{frame}

\begin{frame}
    \frametitle{Other cool stuff}
\begin{itemize}
\item Bottom, partial data and strictness
\item Monad laws
\begin{itemize}
    \item Similar to the monoid laws
    \item Just a monoid in the category of endofunctors after all
\end{itemize}
\item Fold fusion
\begin{itemize}
    \item when does \Verb?f . fold g a = fold h b??
\end{itemize}
\item Program synthesis
\item Theorems for free
\end{itemize}
\end{frame}

\begin{frame}
    \frametitle{Books}
\begin{itemize}
\item How To Prove It - Velleman
\item Introduction to Functional Programming with Haskell - Bird
\item Interactive Theorem Proving and Program Development - Bertot
\item Software Foundations - Pierce
\end{itemize}
\end{frame}

\end{document}
